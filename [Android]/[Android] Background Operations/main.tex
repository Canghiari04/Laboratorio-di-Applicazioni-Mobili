\documentclass{article}

\usepackage[utf8]{inputenc}
\usepackage[T1]{fontenc}
\usepackage{lipsum}
\usepackage{graphicx}
\usepackage{amsmath}
\usepackage[margin=1in]{geometry}
\usepackage{titlesec}
\usepackage{enumitem}
\usepackage{geometry}
\usepackage{tabularx}
\usepackage{caption}
\usepackage{fixltx2e}
\usepackage{booktabs}
\usepackage{float}  
\usepackage{graphicx}
\usepackage{floatflt,epsfig}
\usepackage[margin=1in]{geometry} 
\usepackage{lipsum}
\usepackage{graphicx}
\usepackage{listings}
\usepackage{xcolor}

\lstdefinelanguage{XML}
{
  morestring=[b]",
  morestring=[s]{>}{<},
  morecomment=[s]{<?}{?>},
  morecomment=[s][\color{green!50!black}]{<!--}{-->},
  stringstyle=\color{blue},
  identifierstyle=\color{red},
  keywordstyle=\color{orange},
  commentstyle=\color{green!50!black},
  basicstyle=\small\ttfamily,
  frame=single, 
  breaklines=true,
  breakatwhitespace=true,
  tabsize=2,
  showstringspaces=false,
  captionpos=b,
}

\lstdefinelanguage{Java}{
  keywords={abstract,assert,boolean,break,byte,case,catch,char,class,const,continue,default,do,double,else,enum,extends,false,final,finally,float,for,goto,if,implements,import,instanceof,int,interface,long,native,new,null,package,private,protected,public,return,short,static,strictfp,super,switch,synchronized,this,throw,throws,transient,true,try,void,volatile,while},
  morekeywords={[2]System,out},
  morecomment=[l]{//},
  morecomment=[s]{/*}{*/},
  morestring=[b]",
  basicstyle=\small\ttfamily,
  keywordstyle=\color{blue}\bfseries,
  keywordstyle={[2]\color{orange}\bfseries},
  commentstyle=\color{green!70!black},
  stringstyle=\color{red},
  showstringspaces=false,
  tabsize=2,
  breaklines=true,
  breakatwhitespace=true,
  frame=single, 
  captionpos=b
}

\titleformat{\section}
{\LARGE\bfseries}{\thesection}{1em}{}

\titleformat{\subsection}
{\Large\bfseries}{\thesection}{1em}{}

\begin{document}

\pagestyle{empty}

\section*{Background Operations}
\large

\subsection*{Short Recap}
Fino ad ora, la complessità dell'applicazioni \textit{Android} presentate accomunava un'insieme di \textit{activities}, usufruendo degli \textit{intents} affinchè le stesse possano dialogare, delle \textit{views} per la visualizzazione grafica dell'applicativo ed infine degli \textit{events} per manipolare le interazioni con l'utente.\vspace*{7pt}\\
Tuttavia i concetti, brevemente descritti prima, sono eseguiti in \textit{foreground}; occorre quindi accertarsi di cosa accada al di sotto.     

\subsection*{Notifications}
Le \textbf{notifiche} sono messaggi provenienti dall'applicazione \textit{Android} sviluppata, tipicamente adottati per informare l'utente di certi avvenimenti. Si distinguono in due tipologie, a seconda dell'azione che si voglia conseguire:
\begin{itemize}[label={-}]
  \itemsep0em
  \item \textbf{Informativa passiva}, messaggio disposto a schermo per la sola visualizzazione
  \item \textbf{Informativa attiva}, da cui è possibile aprire l'applicazione oppure eseguire direttamente alcune operazioni
\end{itemize}
Tutte le volte in cui è creata una \textit{notifica} è obbligatorio associarla ad un \textbf{canale comunicativo}, ciò avviene per due motivazioni principali. La prima ragione definisce la volontà dell'utente, stabilendo che tipo di messaggi istantanei voglia visualizzare, mentre la seconda è dettata dalla facilità di controllo mediante le impostazioni di sistema. 
\begin{center}
  \includegraphics*[width=0.7\textwidth]{foto1.png}
\end{center}
L'immagine riportata definisce appropiatamente il meccanismo di funzionamento delle \textit{notifiche}. Di seguito si riportano gli step che caratterizzano il titolo esemplificativo.
\begin{itemize}[label={-}]
  \itemsep0em
  \item \textbf{Notification}, esecuzione della \textit{notifica} tramite l'applicazione, attuando un \textbf{PendingIntent}, ossia un gancio per acquisire la \textit{callback} di riferimento
  \item \textbf{Notification Manager}, componente del sistema \textit{Android} responsabile della manipolazione della \textit{notifica} e della disposizione a schermo della stessa all'interno della \textit{status bar}
  \item \textbf{Status bar}, visualizzazione della \textit{notifica} 
\end{itemize}
Per implementare ed inviare una notifica occorre sottostare a specifici step, cosi definiti.
\begin{itemize}[label={-}]
  \itemsep0em
  \item \textbf{Primo step}, ottenere una referenza circoscritta al \textbf{NotificationManager}, dove in linguaggio \textit{Kotlin} si traduce in:
  \begin{lstlisting}[language=JAVA]
val notificationManager = NotificationManagerCompat.from(this)
  \end{lstlisting}
  In breve, si tratta di un'invocazione del \textit{System Service}, affinchè sia possibile comunicare al \textit{sistema operativo} dell'esecuzione di un'operazione che potrebbe avere delle conseguenze sull'intero dispositivo
  \item \textbf{Secondo step}, creare un \textbf{canale di notifica}, attuato per le medesime motivazioni narrate precedentemente.
  \begin{lstlisting}[language=JAVA]
if (Build.VERSION.SDK_INT >= Build.VERSION_CODES.O) {
  val CHANNEL_ID = "My Channel Id"
  val importance = NotificationManager.IMPORTANCE_DEFAULT
  val channel = NotificationChannel(CHANNEL_ID, "MyChannelName", importance)
  channel.description = "My description"
  val notificationManager = NotificationManagerCompat.from(this)
  notificationManager.createNotificationChannel(channel)
}
  \end{lstlisting}
  \item \textbf{Terzo step}, costruire il corpo del messaggio, mediante l'utilizzo del \textit{design pattern Builder}, in cui è suddivisa la costruzione dell'oggetto dalla propria visualizzazione grafica.
  \begin{lstlisting}[language=JAVA]
val builder = NotificationCompat.Builder(this, CHANNEL_ID)
  .setSmallIcon(androidx.core.R.drawable.notification_bg)
  .setContentTitle("Remember that you will die!")
  .setContentText("Let me explain a number of reasons why this is the case, blah, blah, blah...")
  .setPriority(NotificationCompat.PRIORITY_DEFAULT)
  \end{lstlisting}
  \item \textbf{Quarto step}, attuare il \textit{build} della \textit{notifica} ed eseguirla.
  \begin{lstlisting}[language=JAVA]
val myNotificationId = 0
notificationManager.notify(myNotificationId, builder.build())
  \end{lstlisting}
\end{itemize}
Nonostante, non è stato ancora approfondito il tema relativo a \textit{notifiche} che siano caratterizzate da un \textit{atteggiamento attivo}; ossia la possibilità di eseguire direttamente alcune operazioni dopo aver interagito con le comunicazioni visualizzate nella \textit{status bar}.\vspace*{7pt}\\
A livello di codice è attuato un \textit{Pending Intent}, il quale contiene un apposito \textit{collante} che possa essere lanciato da componenti oppure elementi esterni dall'applicazione sviluppata. 
\begin{lstlisting}[language=JAVA]
val newIntent: Intent = Intent(this, MainActivity.javaClass)
newIntent.flags = Intent.FLAG_ACTIVITY_NEW_TASK or Intent.FLAG_ACTIVITY_CLEAR_TASK
newIntent.putExtra("caller", "notification")
val pendingIntent: PendingIntent = PendingIntent.getActivity (
  this, 0, newIntent, PendingIntent.FLAG_IMMUTABLE
)
\end{lstlisting}
Concludendo è importante sottolineare che tutte le notifiche sono eseguite su molteplici \textit{thread}, gestite dal \textit{sistema operativo}; la volontà di adeguare un meccanismo simile è dettata dalla potenziale possibilità che le \textit{notifiche} possano bloccare il device.

\subsection*{Multithreading}
Un'ottima pratica di programmazione \textit{Android}, prevede di assegnare una qualsiasi operazione, che richieda più di qualche millisecondo prima di terminare, ad uno specifico \textit{thread}. Infatti, tendenzialmente, ogni attività è eseguita sul \textit{main thread} assieme all'\textit{User Interface}; in assenza di una diversificazione, il contesto descritto potrebbe provocare un inadempimento dell'applicazione sviluppata.\vspace*{7pt}\\
Per rispondere a tale necessità, le operazioni accomunate da una caratterizzazione simile a quanto riportato, sono eseguite su \textit{flussi di esecuzione} indipendenti; specificati all'interno del \textit{Manifest} oppure definiti a livello di codice durante il \textit{runtime}. Proseguendo, in questo modo, il \textit{sistema operativo} possiede la totale libertà di gestire e manipolare i molteplici processi \textit{runnati}.\vspace*{7pt}\\
In \textit{Android} esistono tre metodi principali per la creazione di \textit{thread}, suddivisi come segue:
\begin{itemize}[label={-}]
  \itemsep0em
  \item \textbf{Extend Thread}, estendere la classe \textit{Thread}, in cui all'interno del metodo \textit{run()} è riportato il comportamento a cui il \textit{flusso di esecuzione} deve sottostare. Infine, istanziato un oggetto della classe, è richiamata la funzione \textit{start()} per decretarne l'esecuzione.  
  \begin{lstlisting}[language=JAVA]
public class MyThread extends Thread {
  public MyThread() {
    super("My Thread");
  }

  public void run() {
    // do your stuff
  }
}
\end{lstlisting}
\begin{lstlisting}[language=JAVA]
MyThread m = new MyThread();
m.start();
\end{lstlisting}
  \item \textbf{Interface Runnable}, implementare l'interfaccia \textit{Runnable}; tuttavia questo approccio richiede che sia presente un \textbf{Thread Pool}. Un \textit{Thread Pool} è un insieme preallocato di \textit{flussi di esecuzione} che eseguono \textit{task} parallelamente, prelevate da una queue di riferimento.\vspace*{7pt}\\
  In questo modo, nuove mansioni sono eseguite da \textit{thread} esistenti, evitando di inizializzare un numero spropositato di \textit{flussi di esecuzione}.
  \begin{lstlisting}[language=JAVA]
ExecutorService executorService = Executors.newFixedThreadPool(4);
  \end{lstlisting}
\begin{lstlisting}[language=JAVA]
executorService.execute(new Runnable() {
  @Override
  public void run() {
    // do your stuff
  }
});
\end{lstlisting}
\end{itemize}

\subsection*{Message Passing}
\dots

\end{document}